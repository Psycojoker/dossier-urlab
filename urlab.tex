\documentclass{article}
\usepackage[utf8]{inputenc}
\usepackage[french]{babel}
\usepackage[T1]{fontenc}
\usepackage{verbatim}
\usepackage{graphicx}
\usepackage{amsmath}
\usepackage{eurosym}
\usepackage{textcomp}
\usepackage{fullpage}
\usepackage{fix-cm}
\usepackage{textcomp}

\frenchbsetup{ReduceListSpacing=false,CompactItemize=false}

\title{Hackerspace de L'ULB\\ \fontsize{90}{100}\selectfont UrLab}
\author{ }

\begin{document}
\maketitle{}
\newpage
\tableofcontents
\newpage
\setlength{\parskip}{0.5ex plus 0.2ex minus 0.2ex}
\setlength{\parindent}{0pt}

\section{Présentation}

Le hackerspace de l'ULB est un endroit où des étudiants de l'ULB ayant un intérêt 
pour l'informatique, l'électronique ou d'une manière générale la technologie, 
peuvent se rencontrer, partager et collaborer. Il peut être comparé à un laboratoire ouvert.
Les thèmes abordés peuvent avoir un lien direct avec les cours mais pas nécéssairement, le but étant d'explorer d'autres domaines, ou d'en approfondir.

Le hackerspace est composé de trois facettes : 
\begin{itemize}
\item Un atelier, lieu de travail où chacun peut construire quelque chose en profitant 
des installations présentes.
\item Un lieu de socialisation, où des groupes de gens peuvent se rencontrer
et échanger des idées et collaborer sur des projets.
\item L'organisation régulière de conférences, présentations, etc.\end{itemize}

\begin{center}
\includegraphics[height=70mm]{turmlabor.jpg}
\includegraphics[height=70mm]{electrolab.jpg}

A gauche le Turmlabor de l'université de Dresde\footnote{http://www.turmlabor.de}, à droite l'Electrolab à Nanterre\footnote{http://www.electrolab.fr}.
\end{center}

Un hackerspace n'a pas de lien avec le piratage informatique.
De plus, des mesures seront prises afin d'éviter tout danger, notamment la mise en place de règles pour l'emploi du matériel. Les puissances électriques employées sont d'une grandeur électronique et l'outil le plus problématique sera probablement le fer à souder. Avec une bonne politique de prévention des risques, ce local ne serait pas plus dangereux qu'une salle de TP machine.

Il existe déjà plus d'une centaine de hackerspaces\footnote{http://hackerspaces.org/wiki/List\_of\_Hacker\_Spaces} dans le monde, c'est un visage de la communauté DIY (do it yourself) à l'ère de
l'informatique. Cet aspect DIY permet d'avoir un accès facilité à un
matériel important qu'il serait impossible de posséder en tant que
particulier. Le hackerspace est la structure portant la nouvelle génération de hackers.

\begin{center}
\includegraphics[height=65mm]{hacker-generations-divided.png}

Générations de hacker. Source: Taylor\footnote{Taylor,  P.A. (2005). From hackers to hacktivists: speed bumps on the global  superhighway? New Media \& Society, 7(5):625.} Modifié\footnote{http://extreme.ajatukseni.net/tag/hackerspace/}.
\end{center}

Ce projet a, entre autre, le soutien du Président du Conseil d'Administration de l'ULB, du Doyen de la Faculté des Sciences, du Département Informatique, du Cercle Informatique, du Bureau des Étudiants en Sciences, du Bureau des Étudiants Administrateurs ainsi que de plus de 200 étudiants à l'ULB. Globalement, nous proposons de créer une structure non directive favorisant la créativité des étudiants en fournissant la logistique nécessaire pour que les projets aboutissent.

\section{Exemples concrets}

Voici quelques réalisations d'autres hackerspaces qui intéressent déjà nos membres :
\begin{description}
\item[RepRap]\footnote{http://en.wikipedia.org/wiki/Reprap} Une imprimante 3D professionnelle coûte plusieurs milliers d'euros, elle permet notamment de faire du prototypage rapide : c'est un outil de choix dans un atelier. Un projet DIY existe depuis plusieurs années visant à réaliser les plans d'un tel outil pouvant se répliquer lui-même, d'où le nom. Actuellement, elle est façonnable pour environ 500 euros.

\begin{center}
\includegraphics[height=70mm]{reprap-darwin.jpg}

Une RepRap. Source: Wikipedia\footnote{http://upload.wikimedia.org/wikipedia/commons/thumb/f/f8/Reprap\_Darwin.jpg/480px-Reprap\_Darwin.jpg}
\end{center}

\item[LaserCutter]\footnote{http://floridacreatives.com/barcamp/laser-cutter-familab-hackerspace} Toutes les variantes des CNC\footnote{Computer Numerical Control : machine-outil automatisée} open source sont d'autres cibles évidentes pour un laboratoire de ce type.
\item[VendingMachine] Pour organiser la vente des composants électroniques nécessaire à la réalisation de projets sans nécessairement disposer d'une caisse et d'un responsable à tout moment, le NYResistor\footnote{http://www.nycresistor.com/2010/01/22/parts-vending-machine/} (hackerspace de New York) a détourné un distributeur de friandises. Cette VendingMachine est un parfait exemple de réutilisation d'appareil usagé.
\item[LockRFID] La technologie RFID prend une place de plus en plus grande dans nos vies et soulève des questions éthiques évidentes relatives à la protection de la vie privée. L'étude de ces questions pourrait être abordée par exemple en réalisant un système de serrure et de carte RFID ouvrant certaines armoires du hackerspace.
\item[iRail]\footnote{http://www.irail.be/route/} Projet du Whitespace, hackerspace gantois, visant à rendre utilisable les sites publics (surtout celui de la SNCB). Un projet ayant la même philosophie a été réalisé par un étudiant soutenant le projet hackerspace dans le but de rendre utilisable GeHoL.
\item[Visiter le CERN] L'avantage de la structure non directive est qu'il est possible d'organiser des projets immatériels. Par exemple, plusieurs étudiants participants au projet planifient une visite au CERN.
\end{description}

Une partie importante des projets initiaux du hackerspace sera avant tout de fabriquer des outils et de réparer de l'équipement de seconde main pour compléter les ateliers.

\section{Avantages pour l'ULB}

Le hackerspace peut notamment apporter à l'ULB : 
\begin{itemize}
\item Une implication plus importante des étudiants dans la vie active de l'ULB.
Actuellement, en dehors du foklore et des cercles politiques, peu d'initiatives existent afin d'impliquer les étudiants\footnote{L'ULB propose toute une série d'activités culturelles, mais comme un sondage réalisé par le CI parmi les étudiants de BA1 et de BA2 le montre, cela fait partie des activités extra-scolaires les moins populaires.}.
Avec un laboratoire ouvert, nous pensons susciter la curiosité et l'envie d'apprendre, ce qui pourrait diminuer à terme le décrochage universitaire.
\item Une augmentation de la visibilité et du rayonnement de l'ULB dans certains domaines technologiques.
Par exemple, plusieurs universités de part le monde sont connues dans la communauté informatique
principalement grâce au hackerspace qu'elles hébergent.
\item Ce projet présentant des aspects intéressants pour les polytechniciens, les informaticiens,
les architectes ou encore les physiciens, sa mise en place permettra de créer un nouvel endroit d'émulsion
et d'échanges interfacultaires.
\item Ce projet visant à la construction d'équipement pointu, il pourrait être utilisé dans le cadre d'autres projets étudiants par les facultés (par exemple, un travail sur les moteurs pas à pas utilisé par une imprimante 3D).
\end{itemize}

\section{Infrastructure}

Pour fonctionner, le hackerspace a besoin d'un local, d'électricité 
(sur plusieurs circuits si possible), d'eau et d'un accès à Internet. Une indépendance concernant les heures d'ouvertures serait utile.

Par ailleurs, le local sera divisé en deux parties. L'une à but social favorisant la communication, les débats et échanges d'idées. L'autre à but pratique, plus studieuse permettant la réalisation des idées engendrées lors des discussions. 

Dans la partie sociale, il y aura un frigo, une machine à café, un tableau blanc, 
des armoires pour stocker le matériel important.
\newline
Dans la partie pratique se trouveront les équipements électroniques, des composants, des outils et des ordinateurs.
\newline
Dans les deux parties, il faudra des tables et des chaises ainsi que des étagères.

\section{Gestion de l'argent}
\subsection{Financement}

Plusieurs systèmes de financement sont prévus :
\begin{description}
\item[Financement initial :] au quotidien, peu de fonds de roulement sont nécessaires. A contrario, l'investissement de départ sera conséquent afin de lancer le projet.
\item[Cotisation de membres :] annuelle, avec un prix faible et donnant le status de membre. 
Ce statut permet de participer au processus décisionnel. L'objectif de la cotisation est aussi de responsabiliser les membres.
\item[Dons d'argent :] provenants de particuliers ou d'associations voulant soutenir notre action.
\item[Levée de fonds :] pour un projet spécifique (achat de matériel, etc.). C'est 
le même principe que le précédent, sauf que les généreux donnateurs auront la 
garantie que leur argent sera employé dans un but précis.\end{description}

\subsection{Ventes}

Sur place, il sera possible d'acheter des boissons (café, club mate, cola, etc. , mais  
pas de boisson alcolisée) ainsi que des collations (Snickers, Mars, Lion, chips, etc.).
Il sera aussi possible d'acheter des composants électroniques pour les besoins des projets.


\section{Structure}

Le hackerspace est composé d'un mainteneur, d'un conseil d'administration et des membres.
Le mainteneur est délégué au Cercle Informatique de l'ULB, le hackerspace est 
une sous-structure dudit cercle, principalement pour des raisons d'assurance et
de responsabilité, ainsi que pour obtenir le status d'ASBL. Par contre, les deux groupes de 
membres (membres du cercle, membres du hackerspace) seront séparés, étant donné que 
les deux populations sont différentes (et indifférentes à l'existence de l'autre). 
Le mainteneur ainsi que le conseil d'administration seront élus par les 
membres du hackerspace.

\section{Objectifs et méthodes}

Le hackerspace souscrit au principe du Libre examen et aux autres valeurs de l'ULB.
Il souscrit aussi à une idée de Liberté dans un sens plus large en encourageant et en promulguant l'usage de logiciels libres, la neutralité d'Internet, le partage et l'ouverture de 
l'information et plus généralement l'accès universel, libre et gratuit aux connaissances.

Le hackerspace est conscient qu'il présente peu d'intérêt si isolé. Il cherche donc 
à s'inscrire dans une logique de continuité et à s'intégrer aux réseaux d'hackerspaces 
internationaux.

Pour évoluer, il doit notamment connaitre sa propre histoire : il est 
donc très important de disposer d'archives ouvertes et entretenues de manière 
automatique (cf. incubator). Plus spécifiquement, des statistiques seront 
conservées pour mettre en évidence les contributions des membres et leurs conséquences. 
Le hackerspace doit aussi être à l'écoute de tous ses membres et fournira des moyens simples pour 
donner un feedback. 

Pour vivre et s'enrichir harmonieusement, le hackerspace doit disposer d'une base 
règlementaire. Il est évident qu'il respectera le règlement de l'ULB 
(charte horaire, réglementation au niveau de la sécurité ou du tabac). De plus, il souscrit aux 
"design patterns"\footnote{http://hackerspaces.org/wiki/Design\_Patterns}
\footnote{https://hackerspace.be/Patterns} reconnus et plus généralement à une 
éthique\footnote{http://en.wikipedia.org/wiki/Hacker\_ethic}, sauf en cas de 
contradiction avec sa situation spécifique.


\section{Thèmes}

Sujets proposés comme base de travail lors des 
conférences et des workshops. Il s'agit d'une liste non exhaustive et de pistes purement indicatives :
\begin{itemize}
\item Programmation avancée
\item Retroingénierie
\item Administration système et réseau
\item Lifestyle, lifehacking
\item Hardware hacking
\item Electronique et interfaçage avec un ordinateur (Arduino etc.)
\item Radio-identification (technologie RFID)
\end{itemize}

\section{Processus}
\subsection{Incubator}

Il existe trois types de personnes potentiellement interessées par le hackerspace :
\begin{itemize}
\item Les personnes motivées ayant envie de travailler en groupe sur un projet, mais qui n'ont pas d'idée : 
le hackerspace leur permettra de trouver des personnes qui ont, eux, des idées sur lesquelles travailler.
\item Les personnes qui ont une idée précise mais manquent de moyens techniques pour la réaliser : 
le hackerspace mettra à leur disposition les moyens techniques nécessaires.
\item Les personnes qui ont une idée précise mais manquent de moyens humains : 
le hackerspace leur permettra de trouver d'autres personnes intéressées par leur projet et prêtes à les aider.
\end{itemize}

Dans le but de faciliter la mise en commun et d'augmenter la chance qu'un projet 
arrive à terme, il faudra mettre en place un incubator, c'est à dire un service en ligne permettant :
\begin{itemize}
\item De lister l'ensemble des membres motivés avec leurs compétences et leurs 
centres d'intérêts, permettant d'en avoir une vision d'ensemble.
\item De visualiser les projets en cours au sein du hackerspace ainsi que les projets planifiés. C'est à dire de connaître leur but, leur feuille de route mais aussi d'autres informations utiles. 
\item De permettre aux membres d'adhérer à un ensemble de projets pour lesquels ils sont prêt à s'investir.
\item De lancer les projets qui ont atteint un but clairement défini et sur lequel suffisamment de membres sont prêts à travailler.
\item De coordonner la force de travail sur plusieurs projets en cas de mise en pause forcée de l'un de ceux-ci.
\end{itemize}

Cette méthode de travail, en plus d'augmenter la productivité générale, permettra 
aux nouveaux membres de s'intégrer rapidement et facilement à la structure en place tout en diminuant 
le temps de "découverte" du hackerspace.

\subsection{Quibossocratie}

Il s'agit d'une méthode employée par des groupes aux moyens humains très limités : si quelqu'un prend 
une initiative et y travaille, il prend également les décisions relatives au projet. 
Cela part de la constatation que l'immobilisme par peur des conséquences est une 
plaie dans toute structure associative et qu'il est plus facile de demander pardon 
que de demander la permission.

Cette méthode n'implique pas forcément une abscence de dialogue ni d'entente. 
Elle permet surtout de limiter les "il faut que", "il n'y a qu'à" ou "ne devrait-on pas lancer...": just do it.

\subsection{Gestion des membres, heures d'ouverture, prise de décision}

Il y aura des "hacking party" organisées le soir une à deux fois par semaine, dont au moins
une précédée par une réunion.
Le reste du temps, le hackerspace sera ouvert en après-midi et soirée selon les 
besoins et les disponibilités.
Une fois par mois, le hackerspace organisera une conférence.

\newpage
\section{Historique}

Au premier semestre nous étions plusieurs étudiants voulant monter un hackerspace.
Pour ce faire nous nous sommes lancés dans l'organisation de "Linux install party" et
de "coding party" dans le but de réunir des gens motivés pour lancer le projet
autour de ce noyau.

Puis, une rencontre avec Markus Lindström et Stephane Fermandes Medeiros, assistants au DI, nous a convaincu de la possibilité d'obtention d'un local à l'ULB si nous présentions un projet concret et sérieux. Nous avons alors changé de mode
opératoire pour nous tourner vers un projet "infrastructure-driven" (endroit, connectivité, serveurs..),
l'histoire montrant que ceux-ci sont plus viables.  Au départ, les deux
assistants sus-nommés voulaient fonder un laboratoire pédagogique interfacultaire,
dont le hackerspace serait une composante. Maheureusement depuis, ce projet plus
global a été mis en attente.

Fin février, nous avons lancé un site web\footnote{http://hackerspace.cerkinfo.be} comportant
une invitation à signer une pétition de soutien pour notre projet. La nouvelle s'est répandue rapidement parmi les intéressés. Actuellement, cette pétition recueille le soutien de plus de 220 étudiants ainsi que celui d'une dizaine de membres du corps académique.

Depuis, ce projet a obtenu le soutien du président du CA, de la Doyenne de la Faculté des Sciences, du département Informatique, du 
Cercle Informatique ainsi que du Bureau des Etudiants en Sciences et du Bureau des Etudiants Administrateurs.
De plus, plusieurs membres du Bureau des Etudiants de Polytechnique ainsi que de certains services de la Faculté
des Sciences Appliquées (CoDE et OPERA notament) ont également apporté leur soutien.

\section{Conclusion}
Nous pensons que l'ouverture d'un hackerspace à l'ULB serait une véritable opportunité pour notre alma mater. En effet, sa création comblerait un véritable manque de structure créées par les étudiants et pour les étudiants désirant s'investir et mettre en pratique les connaissances acquises lors de leur cursus. Véritable lieu de partage, d'échange et d'émulation pour les passionnés, en plus de tous les avantages qu'une structure de ce genre apporterait aux étudiants, il pourrait être un rayonnement supplémentaire pour l'université dans les milieux spécialisés.

Nous avons la conviction qu'avec ce projet, notre Université ne pourrait qu'en ressortir grandie.


\end{document}
