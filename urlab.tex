\documentclass[a4paper]{article}
\usepackage[utf8]{inputenc}
\usepackage[T1]{fontenc} 
\usepackage{geometry}  
\usepackage[francais]{babel} 
\title{UrLaB}  % ajouter un sous titre du genre : Laboratoire étudiant de l'ULB
%\author{...}
% \date{}
\begin{document}
\maketitle
\newpage
\tableofcontents 
\newpage
\section{Introduction}
Le  \emph{UrLaB} est un projet de laboratoire étudiant au sein de l'ULB,  respectant les concepts et principes d'un \emph{Hacker Space}  \footnote{Voir section suivante pour de plus amples explications sur ce  concept.} tel que déjà existant dans d'autres grandes universités telles  que le MIT, ... . % en trouver d'autres 
C'est  un endroit où des étudiants de l’ULB ayant un intérêt pour  l’informatique, l’électronique ou d’une manière générale la  technologie, peuvent se rencontrer, partager et collaborer. 
Comme son nom l'indique (\emph{UrLaB} vient de \emph{Your Lab}), il peut être comparé à un laboratoire ouvert à tous.
 Les  thèmes abordés peuvent avoir un lien direct avec les cours mais pas  néssairement, le but étant d’explorer d’autres domaines, ou d’en  approfondir.

\section{Qu'est ce qu'un hackerspace ?}

Un \emph{hackerspace}/footnote{Source : Wikipedia : http://fr.wikipedia.org/wiki/Hackerspace }, \emph{hacklab} ou \emph{media hacklab}  est un lieu où des personnes ayant un intérêt commun (souvent dans le  domaine de l'informatique, de la technologie, des sciences, des arts...)  peuvent se rencontrer et collaborer. Les Hackerspaces peuvent être vus  comme des laboratoires communautaires ouverts où tous peuvent partager  ressources et savoir. Beaucoup de hackerspaces utilisent et participent à  des projets de logiciels libres, du hardware libre, ou des médias  alternatifs.
Ils sont le plus part du temps physiquement installés dans des maisons d'associations ou d'universités, mais dès que le nombre d'adhérents et  l'éventail des activités augmente ils déménagent généralement dans des  espaces industriels ou d'anciens entrepôts.

\begin{list}{-}{} Un hackerspace est composé de trois facettes :
\item Un workshop, lieu de travail où chacun peut mener à bien un projet en profitant des installations présentes;
\item Un lieu de socialisation, de rencontre, d\rq{}échanges d'idées et de collaboration sur des projets;
\item L’organisation régulière de conférences, présentations par des membres ou des non membres.
\end{list}


        \subsection*{Origine} % pas top cette sous section, à revoir ...
         Il existe actuellement déjà plus d’une centaine de hackerspaces dans le monde.
         C’est un visage de la communauté DIY (do it yourself) à l’ère de  l’informatique. 
Cet aspect DIY permet d’avoir un accès facilité à un  matériel important qu’il serait impossible d’avoir en tant que  particulier. 
La mouvance hacker liée à cette communauté est  responsable notamment du world wide web et du logiciel libre qui le fait  tourner. 
Le hackerspace est la structure portant la nouvelle  génération de hackers actuellement.
% de quel type de hacker tu parle la? ca me semble pas super clair--

        \subsection*{Exemples concrets de réalisation}
Voici quelques réalisations d’autres hackerspaces qui intéressent déjà nos membres :

\paragraph{RepRap 5} Une  imprimante 3D professionnelle coûte plusieurs milliers d’euro, elle  permet notamment de faire du ’rapid prototyping’ : c’est un outil de  choix dans un atelier. Un projet DIY existe depuis plusieurs années  visant à faire les plans d’un tel outil pouvant se répliquer  lui-même, d’où le nom. Actuellement, elle est façonnable pour environ  500 euro.

\paragraph{LaserCuter 7} Et toutes les variantes des CNC homemade sont d’autres cibles évidentes pour un laboratoire de ce type.

\paragraph{VendingMachine}  Pour organiser la vente des composants électroniques nécessaire à la  réalisation de projet sans nécessairement avoir une caisse et un  responsable à tout moment, le NYResistor8:  (hackerspace de New York) a détourné un distributeur de friandises.  Cette vending machine est un parfait exemple de réutilisation  d’appareil usagé.

\paragraph{LockRFID}  La technologie RFID prend une place de plus en plus grande dans nos  vies et soulève des questions éthiques relatives à la protection de  la vie privée évidentes. L’étude de ces questions pourrait être  abordée par exemple en réalisant un système de serrure et de carte  RFID ouvrant certaines armoires du hackerspace.

\paragraph{iRail 9}  Projet du Whitespace, hackerspace gantois, visant à rendre utilisable  les sites publics (surtout celui de la SNCB). Un projet ayant la même  philosophie a été réalisé par un étudiant portant le projet  hackerspace dans le but de rendre utilisable GeHoL.

\paragraph{Visiter le CERN}  L’avantage de la structure non directive est qu’il est possible  d’organiser des projets immatériels, par exemple plusieurs étudiants  du projet planifient une visite au CERN.

Une  partie importante des projets initiaux du hackerspace sera avant tout  de fabriquer des outils et de réparer de l’équipement de seconde main  pour compléter le workshop.


        \subsection*{Préjugés}
        Contrairement aux idées préconcues, un hackerspace n’a aucun lien avec le piratage informatique,  en effet, le therme hacker est ici pris dans son sens premier, c'est à  dire 'bricoleur', c'est donc un atelier d'entraide étudiants. 

\begin{list}{-}{}De  plus, des mesures seront prises afin d’éviter tout danger notamment la  mise en place de règles strictes pour l’emploi du matériel : 
\item Limitation des puissances électriques utilisées à celle de l'électronique de signal (c'est à dire de l'ordre du mWatt);
\item Une politique de prévention des risques par un encadrement et des formations pour l'utilisation des outils à risque;
\item Des regles d'utilisations strictes à respecter par chacuns. % faire un référence vers une pseudo charte d'utilisation du matos ....
\end{list}
Avec ces mesures et la conscientisation des utilisateurs, ce local sera plus sûr qu'une qu’une salle de TP machine.



\section{Objectifs et principes de l'UrLaB}
L'\emph{UrLaB} est  un endroit où des étudiants de l’ULB ayant un intérêt pour  l’informatique, l’électronique ou d’une manière générale la  technologie, peuvent se rencontrer, partager et collaborer. 
Comme  son nom l'indique (\emph{UrLaB} vient de \emph{Your Lab}), il peut être  comparé à un laboratoire ouvert. Globalement, nous proposons de créer  une structure non directive favorisant la créativité des étudiants en  fournissant la logistique nécessaire pour que les projets aboutissent.

% TODO : Reformuler   ---+/- organisation au jour le jour
Il y aura une ou deux grosses soirées par semaine de rassemblement pour avancer sur les projets,  dont au moins une précédée par une réunion. Le reste du temps, le  hackerspace sera ouvert en après-midi et soirée selon les besoins et  les disponibilités. De plus, une fois par mois, une conférence sera organisée par les membres, ou éventuellement des externes désireux de partager leurs connaissances.
Le  hackerspace souscrit au principe du libre examen et aux autres valeurs  de l’ULB. Il souscrit aussi à une idée de liberté dans un sens plus  large en encourageant et en promulguant l’usage de logiciels libres, la  neutralité d’Internet, le partage et l’ouverture de l’information et  plus généralement l’accès universel, libre et gratuit aux  connaissances.
Toutefois, il présente peu d’intérêt si isolé, il se doit donc de s’inscrire dans une logique de continuité et à s’intégrer aux réseaux d’hackerspaces internationaux.
Le  hackerspace, pour évoluer, doit notamment connaitre sa propre histoire  : il est donc très important de disposer d’archives ouvertes et  entretenues de manière automatique (cf. incubator). Plus  spécifiquement, des statistiques seront conservées pour mettre en  évidence les contributions des membres et leurs conséquences. Il doit  aussi être à l’écoute de tous ses membres et fournira des moyens  simples pour donner un feedback. Pour vivre et s’enrichir  harmonieusement, une base règlementaire doit également être mise sur pied.  Il est évident que le hackerspace respectera les règles de l’ULB  (charte horaire, réglementation au niveau de la sécurité ou du  tabac). De plus, il souscrit aux design patterns11 12 reconnus et plus  généralement à une éthique13, sauf si c’est en contradiction avec sa  situation spécifique.

        \subsection*{Public visé}
\begin{list}{-}{} Il existe trois catégorie de personnes potentiellement interessées par l'\emph{UrLaB} :
\item Une personne ayant l'envie de travailler en groupe sur un projet, mais sans avoir d'idée;
\item Une personne ayant une idée précise mais manquant de moyen technique, l'\emph{UrLaB} essaye de répondre à ses besoins dans la mesure du possible;
\item Une personne ayant un idée précise mais manquant de moyen humain : la première catégorie peut répondre à ce besoin.
\end{list}

        \subsection*{Incubateur} % TODO
Dans  le but de faciliter la mise en commun et d’augmenter la chance qu’un  projet arrive à terme, il faudra mettre en place un incubator, un  service en ligne permettant :
–  de lister l’ensemble des membres motivés avec leurs compétences et  leurs centres d’intérêts, pour avoir une vision d’ensemble
– de visualiser les projets (définition, explications, roadmap, info utiles...) en cours et en planification
– de permettre à tous d’adhérer à un ensemble de projets auquel il est prêt à s’investir.
– de lancer les projets qui ont atteint un seuil critique de définition et de gens prêts à travailler dessus
– de coordonner la workforce sur plusieurs projets en cas de mise en pause forcée d’un autre
Cette  méthode de travail, en plus d’augmenter la productivité générale,  permettra à des gens de rapidement s’intégrer facilement à la  structure en place


        \subsection*{Quibossocratie} % TODO
Méthode  employée par des groupes au moyen humain très limité : si quelqu’un  prend une initiative, travaille, c’est lui qui prend les décisions  relatives au projet. Cela part de la constatation que l’immobilisme par  peur des conséquences est une plaie dans toute structure associative et  qu’il est plus facile de demander pardon que de demander la permission.
Cette  méthode n’implique pas forcément une abscence de dialogue ni  d’entente. C’est juste pour limiter les "il faut que", "Il n’y a qu’à",  "on ne devrait pas lancer..." : Just do it.

        \subsection*{Charte} %TODO :  En résumer les principes globaux et traduire/adapter celui-ci à mettre en annexe ( ou mettre ici si assez de place)

source : http://hackerspaces.org/wiki/Release_of_Liability

In consideration for my being permitted to participate in the activities of XXX, I agree to the following waiver and release:
ASSUMPTION OF RISK:  I acknowledge that inherent risks, dangers and hazards and such exist  when using power tools commonly used in electronics construction,  fabrication, software design and other technology related activities.  Participation in such activities and/or the use of equipment associated  with technology design, manufacture and experimentation may result in  injury, illness, death or damage to personal property. These risks and  dangers may be caused by other participants, members or by accidents,  acts of nature or other causes. Risks and dangers may arise from  foreseeable or unforeseeable causes including, but not limited to  electrocution, burns, impalement, injury from slips or falls, etc.
RELEASE OF LIABILITY:  The member fully assumes all risks associated with participation in  events and exempts and releases XXX, its members, officers, agents,  board members, from action whatsoever arising out of any damage, loss or  injury to the participant or the participant’s property while upon the  premises or using any equipment of the organization or while  participating in any of the activities contemplated by this agreement  whether such loss, damage, or injury results from the negligence of the  corporation, its members, agents, or from some other cause. 
COVENANT NOT TO SUE:  The participant agrees never to institute any suit or action at law  otherwise against XXX, its members, officers, board members, agents, nor  to initiate or any way assist the prosecution of any claim for damages  or course of action which the member, member’s heirs, executors or  administrators hereafter may have by reason of injury to the person of  the member or to the participant’s property arising from the activities  contemplated by this agreement.
THIRD PARTY INDEMNIFICATION:  The member will indemnify, save and hold harmless XXX, its members,  officers, board members, or agents from any and all losses, claims,  actions, or proceedings of every kind and character which may be  presented or initiated by any other persons or organizations and which  arise directly or indirectly from the actions of the member while  engaged in the activities contemplated by this agreement.
I  hereby acknowledge that I have CAREFULLY read all of the provisions  above, fully understand the terms and conditions expressed there, and do  freely choose acceptance of the provisions of the foregoing paragraphs  relating to assumption of risk, release of liability, covenant not to  sue, and third party indemnification.

         \subsection*{Thèmes} % TODO : petit explication en deux mot de ce que c'est chaque fois
Différentes  idées de sujets ont déjà été proposés pour être pris comme base de  travail pour les workshop et conférences (liste non exhaustive) : 
\begin{list}{-}{}
\item Programmation avancée
\item Administration système et réseau
\item Lifestyle, lifehacking
\item Hardware hacking
\item Electronique et interfaçage avec un ordinateur (arduino..) 
\item Radio-identification (technologie RFID)
\end{list}


\section{Gestion/structure administrative ?} %TODO : Décrire le role de chacun de ces organes/ personnes
L’\emph{UrLaB}  est composé d’un mainteneur, d’un conseil d’administration et des  membres. Le mainteneur est délégué au Cercle Informatique de l’ULB.
Pour  des raisons d’assurance, de responsabilité et afin d’avoir le status  d’ASBL, l’\emph{UrLaB} est une sous-structure du-dit cercle. 
Les  deux entités sont et resterons indépendantes et, l’élection du  mainteneur et du conseil d’administration sont réalisées uniquement  par les membres de l’\emph{UrLaB}. Cette séparation n\rq{}empéchant en  rien les collaborations et la double implication des membres (sans toute  fois imposer cette dernière).


\section{Infrastructure}
Pour  fonctionner, l’\emph{UrLaB} a besoin d’un local, d’électricité (sur  plusieurs circuits si possible), d’eau et d’une connection internet. Une  relative indépendance au niveau des heures d’ouvertures serait utile,  par exemple lors de l’organisation de conférences.
\begin{list}{-}{} Par ailleurs, le local sera divisé en deux parties :
\item  La première à but pratique, plus studieuse et permettant la  réalisation pratique des idées engendrées lors des discussions. 
\item La seconde à un but social, permettant de favoriser la communication, les débats et échanges d’idées.
\end{list}
Dans  la salle sociale, il y aura principalement un frigo, une machine à  café, un tableau blanc, des armoires pour stocker le matériel  important... 
Dans les deux parties, il faudra évidement des tables, des chaises ainsi que des étagères.


\section{Aspect financier} %TODO : reformuler 
Plusieurs systèmes de financement sont prévus :
\begin{list}{o}{}
\item Financement initial qui sera assez conséquent, affin d'obtenir tout les bien nécessaires au bon fonctionnement du projet.
\item Au quotidien, peu de fonds de roulement sont nécessaires.
\item Cotisation  de membres annuelle, avec un prix faible, donnant le status de membre.  Ce statut permet de participer au processus décisionnel et d’ouvrir le  hackerspace. L’objectif de la cotisation est aussi de responsabiliser  les membres.
\item Dons d’argent de particuliers ou d’associations voulant soutenir notre action.
\item Levée de fonds pour un projet spécifique (achat de matériel...) C’est le même principe que le précédent,
sauf que les généreux donnateurs auront la garantie que leur argent sera employé dans un but précis.
\item Sur  place, il sera possible d’acheter des boissons (café, club mate,  coca... - pas de boisson alcolisée) ainsi que des collations (snickers,  mars, lion, chips..). Il sera aussi possible d’acheter des composants  électroniques pour les besoins des projets.
\end{list} 


\section{Historique du projet} % sous forme plus chronologique peut etre ? un paragraphe par périodesignificative ?
Au  premier semestre nous étions plusieurs étudiants voulant monter un  hackerspace. Pour ce faire nous nous sommes lancés dans l’organisation  de linux install party et de coding party dans le but de réunir des  gens motivés pour lancer le projet autour autour de ce noyau. --- Commence cela s'est t'il terminé ---
Puis,  suite à une rencontre avec Markus Lindström et Stephane Fermandes  Medeiros, assistants au DI, ces derniers nous ont convaincus de la  faisabilité d’avoir un local à l’ULB si nous présentions un projet  concret et sérieux. Nous avons donc changé  de mode opératoire pour en faire un projet infrastructure-driven  (endroit, connectivité, serveurs..), l’histoire montrant que cela rend  ce type de projet plus viable. Au départ, les deux assistants  sus-nommés voulaient fonder un laboratoire pédagogique  interfacultaire, dont le hackerspace serait une composante.  Maheureusement depuis, ce projet plus global a été mis en attente.
Fin  février, nous avons lancé un site web14 comportant une invitation à  signer une pétition de soutien pour notre projet. La nouvelle s’est  répandue rapidement parmi les intéressés; actuellement, cette  pétition recueille le soutien de plus de 300 étudiants ainsi que celui d’une dizaine de membres du corps académique.
Depuis lors,  ce projet a obtenu le soutien du président du CA, de la Doyenne de la  Faculté des Sciences, du département informatique,du cercle  informatique ainsi que du bureau des étudiants en sciences et du bureau  des étudiants administrateurs. De plus, plusieurs membres du BEP ainsi  que de certains services de la facultés des sciences appliquées  (CoDE, OPERA notament) ont également apporté leur soutien personnel.


\section{Bénéfices pour l'ULB} % TODO 
Le hackerspace peut notamment apporter à l’ULB :
\begin{list}{-}{} 
\item  Implication plus importante des étudiants dans la vie active de l’ULB.  Actuellement, en dehors du foklore et des cercles politiques peu  d’initiatives existent afin d’impliquer les étudiants10, avec un  laboratoire ouvert aux étudiants, nous pensons que cela diminuera le  décrochage scolaire en suscitant la curiosité et l’envie d’apprendre.
\item  Augmentation de la visibilité et du rayonnement de l’ULB dans certains  domaines technologiques, Par exemple, plusieurs universités de part le  monde sont connues principalement grâce au hackerspace qu’elles  hébergent, dans la communauté informatique.
\item  Ce projet présentant des aspects intéressants pour les  polytechniciens, les informaticiens, les architectes ou encore les  physiciens, sa mise en place permettra de créer un nouvel endroit  d’émulation et d’échanges interfacultaires.
\item  Ce projet visant à la construction d’équipement pointu, il pourrait  être utilisé dans le cadre d’autres projets étudiants par les  facultés (par exemple, un travail sur les moteurs pas à pas sur une  imprimante 3D)
\end{list}  


\section{Soutient recu de la communauté universitaire}
\begin{list}{-}{} Ce projet a, entre autre, recu le soutien officiel de : 
\item Le Président du Conseil d’Administration de l’ULB
\item La Doyenne de la Faculté des Sciences
\item Le Département Informatique
\item Le Cercle Informatique
\item Le Bureau des Étudiants en Sciences
\item Le Bureau des Étudiants Administrateurs
\item Plus de 300 étudiants de l’ULB. 
\end{list}


\section{Conclusion} % TODO
Nous  pensons que l’ouverture d’un hackerspace à l’ULB serait une véritable  opportunité pour notre alma mater. En effet, sa création comblerait  un véritable manque de structure par les étudiants pour les étudiants  désirant s’investir et mettre en pratique les connaissances acquises  lors de leur cursus. Véritable lieu de partage, d’échange et  d’émulation pour les passionnés, en plus de tous les avantages qu’une  structure du genre apporterait aux étudiants, il pourrait être un  rayonnement supplémentaire pour l’université dans les milieux  spécialisés.
Nous avons la conviction qu’avec ce projet, notre Université ne pourrait qu’en ressortir grandie.
\end{document}

